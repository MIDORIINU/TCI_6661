
\subsection{Resumen de objetivos}


\normalfont

Diseñar un inversor CMOS en tecnología $90 \si[per-mode=symbol]{\nano\meter}$. Asumir en todos los casos $L = L_{MIN}$ y $W_{P} = 2W_{N}$ y $V_{DD} = 1.2 \si[per-mode=symbol]{\volt}$. Cada estudiante debe utilizar los parámetros indicados en la grilla de asignación de temas.
\\\\


Parámetros asignados (Usuario \textbf{UE07}):


\begin{equation*}
\mathcolorbox{Lightblue}{ W_{N} = 0.92 \si[per-mode=symbol]{\micro\meter} }
\end{equation*}

\begin{equation*}
\mathcolorbox{Lightblue}{ Pitch = 5 \si[per-mode=symbol]{\micro\meter} }
\end{equation*}


\subsection{Desarrollo}


El desarrollo consiste de dos partes, una parte de cálculo manual, punto \num{1}, y una parte de diseño y simulación para validar estos cálculos, puntos \num{2} y \num{3}. Los cálculos se hicieron con referencia al libro de Rabaey~\cite{Rabaey2}.


%\begin{figure}[H] %htb
%\begin{center}
%\includegraphics[width=1.0 \textwidth, angle=0]{./img/enunciado/circuito_enunciado.png}
%\caption{\label{fig:fig_original_circuit}\footnotesize{Circuito propuesto.}}
%\end{center}
%\end{figure}


\clearpage
